\newpage
\subsection{Technische Spezifikation}
Dieser Abschnitt behandelt die technischen Anforderungen des Projekts, die für die Umsetzung der verschiedenen Lastenfälle erforderlich sind. Es wird beschrieben, welche Hardware für jeden Lastenfall benötigt wird und aus welchen Gründen diese Auswahl getroffen wurde. Zudem wird die Wahl bestimmter Standards und Protokolle erläutert, die im Projekt verwendet werden.

\subsubsection{Hardware}




\textbf{Komponenten\hyperlink{LF01_Link}{/LF01/}} \\

\begin{wrapfigure}{r}{0.4\textwidth} % Increase the width of the figure environment
	\vspace{-60pt + 0.02\textwidth}
	\hspace{0.07\textwidth} % Add horizontal space
	\includegraphics[width=0.2\textwidth]{images/05_technische_spezifikation/Interface/Encoder.png} % Keep the image size the same
	\caption{Rotary-Encoder}
	\label{fig:rotary_encoder}
	\vspace{-100pt}
\end{wrapfigure}

\textbf{\hypertarget{Encoder}{Encoder}} \\

\textbf{Model} RIC11-22S16D5M-TH

\textbf{Spannung} 3.3V

\textbf{Mechanisch:}
\begin{itemize}
	\item 20 Impulse pro Umdrehung (20 PPR)
	\item 20 Positionen (20 DET)
	\item Schalter (SW)
\end{itemize} 



Zur Auswahl der Samples und zum Navigieren durch das Menü wir ein Rotary-Encoder mit einem Switch Button eingesetzt.Das Empfangen der \textbf{A} und \textbf{B} Signale des Encoders erfolgt über die Pins \boldinline{PA0} und \boldinline{PA1}.
Das Drücken des Switch über den Pin \boldinline{PA4}. Durch das drücken des Switch-Buttons wird der Pin \boldinline{PA4} auf high gesetzt.

\begin{itemize}
	\item \textbf{Präzise Steuerung:} Der Encoder ermöglicht eine präzise Steuerung des Cursors auf dem Display.
	
	\item \textbf{Benutzerfreundlichkeit:} Der Benutzer kann durch die Liste navigieren und ein Sample auswählen. Die Kombination aus Drehbewegung und Druckknopf-Funktionalität macht den Encoder intuitiv.  
\end{itemize}

\vspace{3em}

\begin{wrapfigure}{r}{0.4\textwidth} % Increase the width of the figure environment
	\vspace{-20pt + 0.02\textwidth}
	\hspace{0.06\textwidth} % Add horizontal space
	\includegraphics[width=0.25\textwidth]{images/05_technische_spezifikation/Interface/Display.png} % Keep the image size the same
	\caption{LCD-Display}
	\label{fig:lcd_display}
	\vspace{-50pt}
\end{wrapfigure}

\textbf{\hypertarget{Display}{LCD-Display}} \\

\textbf{Model:} GME128128-01-ii2

\textbf{Treiber:} SH1107

\textbf{Mode} Monochrom (1Bit)

\textbf{Spannung} 5.0V \\ \\



Zur Visualiesierung der Sampels haben wir einen Monochronen LCD-Display benutzt. Im zusammenspiel mit dem Encoder ermöglicht es eine gute Navigation durch den gewünschten Samplepool. Die Daten werden über den Output Pin  \boldinline{PB7} an den SDA des Displays übertragen. Der Takt wird über den Pin  \boldinline{PB6} an den SCL übertragen. Das Display wird mit 20 FPS betrieben und mit hilfe von \boldinline{Timer tim5} geupdated.

\begin{itemize}
	\item \textbf{Klare Visualisierung:} LCD-Displays bieten eine klare und gut lesbare Darstellung von Text und Grafiken.
	\item \textbf{Anpassbarkeit:} Sie können einfach an verschiedene Layouts und Designs angepasst werden.
\end{itemize}

\newpage
\textbf{Komponenten\hyperlink{LF02_Link}{/LF02/}} \\

\textbf{\hypertarget{Potentiometer}{Schiebe-Potentiometer}}\\

\textbf{Model} Bourns PTL45-15R0-103B2

\textbf{Wiederstand} 10k Ohm

\textbf{Weg} 45mm

\textbf{Spannung} 3.3V \\ \\ \\

	\begin{wrapfigure}{r}{0.4\textwidth} % Increase the width of the figure environment
	\vspace{-155pt + 0.02\textwidth}
	\hspace{0.07\textwidth} % Add horizontal space
	\includegraphics[width=0.2\textwidth]{images/05_technische_spezifikation/Interface/Potentiometer.png} % Keep the image size the same
	\caption{Potentiometer}
	\label{fig:schiebe_potentiometer}
	\vspace{-20pt}
\end{wrapfigure}

Für die Filterfunktion benötigen wir 5 Potentiometer. Es wird zyklisch die Ausgangsspannung des Schleifers abgegriffen die die Teilspannung zwichen den VCC und dem GND darstellt. Dies erfolgt mit Hilfe vom ADC und dem DMA. Die Auswertung der Spannung erfolgt über die Pins  \boldinline{PA6, PA7, PB0, PB1, PC0}. Die Pins  \boldinline{PA6, PA7, PB0, PB1} sind für die Klassen zuständig  \boldinline{PC0} für den Schwellenwert an erlaubter Abweichung.

\begin{itemize}
	\item \textbf{Präzise Steuerung und feine Abstimmung:} Ein Potentiometer ermöglicht eine stufenlose und präzise Einstellung. Durch das Schieben des Potentiometers kann der Benutzer den Cursor in kleinen, genauen Schritten bewegen.
	\item \textbf{Einfache Bedienung und intuitive Nutzung:} Potentiometer sind einfach und intuitiv zu bedienen.
	\item \textbf{Direkte visuelle Rückmeldung:} Durch die sofortige visuelle Rückmeldung auf dem LCD-Display kann der Benutzer sofort sehen, wie sich die Bewegung des Potentiometers auf die Position des Cursors auswirkt.
\end{itemize}


\textbf{(LCD-Display)}\\

Der LCD-Display ist der gleiche wie im \boldinline{/LF01/} beschrieben. Dieser dient zur Darstellung der Fader Einstellung in prozentualer Form.

\newpage

\subsubsection{Pinout Interface Komponente}

\textbf{Interface}

\begin{longtable}[c]{|p{2.5cm}|p{1cm}|p{2.5cm}|p{2.5cm}|p{2.5cm}|p{3cm}|}
	\hline
	\textbf{Komponente} & \textbf{PIN} & \textbf{Signal-On-PIN} &  \textbf{GPIO-Mode} & \textbf{GPIO-Pull-Up/Pull-Down } & \textbf{User-Label}\\
	\hline
	Encoder Menü & PA0 & n/a & EIMRETD & PULL UP & enc\_a\_clk\_in1 \\
	\hline
	& PA1 & n/a &  INPUT & PULL UP & enc\_a\_dt\_in2 \\
	\hline
	& PA4 & n/a & EIMRETD & PULL UP & enc\_a\_switch\_in3 \\
	\hline
	ADC & PA6 & ADC1\_IN6 & ANALOG & NPU NPD & FADER1 \\
	\hline
	& PA7 & ADC1\_IN7 & ANALOG & NPU NPD & FADER2 \\
	\hline
	& PB0 & ADC1\_IN8 & ANALOG & NPU NPD & FADER3 \\
	\hline
	& PB1 & ADC1\_IN9 & ANALOG & NPU NPD & FADER4 \\
	\hline
	& PC0 & ADC1\_IN10 & ANALOG & NPU NPD & FADER5 \\	
	\hline
	I2C & PB6 &I2C1\_SCL & AFOD & PULL UP & n/a \\
	\hline
	& PB7 &I2C1\_SDA & AFOD & PULL UP & n/a \\
	\hline
	SDIO & PC8 & SDIO\_D0 & AFPP & PULL UP & n/a \\
	\hline
	& PC9 & SDIO\_D1 & AFPP & PULL UP & n/a \\
	\hline
	& PC10 & SDIO\_D2 & AFPP & PULL UP & n/a \\
	\hline
	& PC11 & SDIO\_D3 & AFPP & PULL UP & n/a \\
	\hline
	& PC12 & SDIO\_CK & AFPP & NPU NPD & n/a \\
	\hline
	& PD2 & SDIO\_CMD & AFPP & PULL UP & n/a \\
	\hline
\end{longtable}

\textbf{EIMRETD} = External Interrupt Mode and Rising Edge Trigger Detection 		

\textbf{AFOD} = Alternate Function Open Drain

\textbf{NPU NPD} = No Pull Up No Pull Down		

\textbf{AFPP} = Alternate Function Push Pull


\newpage



\paragraph{PCM5102a Breakout Board}\mbox{}\\

\textbf{\hyperlink{lf-audiorecord}{/LF04/}, \hyperlink{lf-audioplayback}{/LF05/}, \hyperlink{lf-pitchaudio}{/LF06/}} \\

\begin{wrapfigure}{r}{0.4\textwidth} % Increase the width of the figure environment
	\vspace{-20pt}
	\includegraphics[width=0.2\textwidth]{images/05_technische_spezifikation/audio/pcm5102a_breakout.jpg}
	\caption{PCM5102a Breakout Board}
	\label{fig:pcm5102a_breakout}
\end{wrapfigure}


Der PCM5102a Audio Codec wurde für die DAC Wandlung der Audiodaten verwendet. Für die Implementation und die Entwicklung des Prototyps wurde ein Breakout Board gekauft, um ohne die Notwendikeit eines PCB-Design die Firmware für \textbf{I M A} zu schreiben. 
Komfortablerweise ist direkt eine \SI{3,5}{\milli\meter} Stereo-Klinkenbuchse auf dem Board verbaut.
Der Pegel beträgt Line-Level (\SI{\pm 1.7}{\volt_{rms}}), und ist somit grob kompatibel zum Eurorack-Standard.

Das Breakout Board wird mit einer Versorgungsspannung von \SI{3,3}{\volt} betrieben.

Als Übertragungsprotokoll dient \textbf{I2S}. Die detaillierte Inbetriebnahme findet sich in Abschnitt \ref{sec:pcm5102a-und-i2s}.

\vspace{4em}

\paragraph{Waveshare Micro SD-Card Module}\mbox{}\\

\textbf{{\hyperlink{LF01_Link}{/LF01/}, \hyperlink{LF02_Link}{/LF02/}, \hyperlink{lf-nn-01}{/LF03/}, \hyperlink{lf-audiorecord}{/LF04/}, \hyperlink{lf-audioplayback}{/LF05/}, \hyperlink{lf-pitchaudio}{/LF06/}}} \\

\begin{wrapfigure}{r}{0.4\textwidth} % Increase the width of the figure environment
	\vspace{-20pt}
	\includegraphics[width=0.4\textwidth]{images/05_technische_spezifikation/audio/waveshare_micro_sd_module.jpg}
	\caption{Waveshare Micro SD-Karten Modul}
	\label{fig:waveshare_micro_sd_module}
\end{wrapfigure}

Das Waveshare Micro SD-Card Module wird für die Speicherung und den Zugriff auf Audiodaten in diesem Projekt verwendet. Dieses Modul ermöglicht die einfache Integration von Micro SD-Karten in das System, um persistente Daten zu lesen und zu schreiben, ohne aufwändige PCB-Designs oder zusätzliche Hardwarekomponenten implementieren zu müssen.

Das Modul unterstützt den Standard \textbf{SDIO}-Bus zur Datenübertragung, was eine einfache Anbindung an das Mikrocontroller-System ermöglicht. 

Die Inbetriebnahme und die spezifischen Konfigurationen für die Verwendung dieses Moduls werden in Abschnitt \ref{sec:sd-card-audio} detailliert beschrieben. 

Das Modul wird mit einer Versorgungsspannung von \SI{3.3}{\volt} betrieben.

\newpage
\subsubsection{Pinout Audio Komponente}

\textbf{Audio}
\begin{longtable}[c]{|p{2.5cm}|p{1cm}|p{2.5cm}|p{2.5cm}|p{2.5cm}|p{3cm}|}
	\hline
	\textbf{Komponente} & \textbf{PIN} & \textbf{Signal-On-PIN} &  \textbf{GPIO-Mode} & \textbf{GPIO-Pull-Up/Pull-Down } & \textbf{User-Label}\\
	\hline
	SDIO & PC8 & SDIO\_D0 & AFPP & PULL UP & n/a \\
	\hline
	& PC9 & SDIO\_D1 & AFPP & PULL UP & n/a \\
	\hline
	& PC10 & SDIO\_D2 & AFPP & PULL UP & n/a \\
	\hline
	& PC11 & SDIO\_D3 & AFPP & PULL UP & n/a \\
	\hline
	& PC12 & SDIO\_CK & AFPP & NPU NPD & n/a \\
	\hline
	& PD2 & SDIO\_CMD & AFPP & PULL UP & n/a \\
	\hline
	I2S & PB10 & IS2\_CK & AFPP  & NPU NPD & n/a \\
	\hline
	& PB12 & IS2\_WS & AFPP & NPU NPD & n/a \\
	\hline
	& PC3 & IS2\_SD & AFPP & NPU NPD &  n/a \\
	\hline
	& PC6 & IS2\_MCK & AFPP & NPU NPD &  n/a \\
	\hline
\end{longtable}

\textbf{NPU NPD} = No Pull Up No Pull Down		

\textbf{AFPP} = Alternate Function Push Pull 



\newpage
\paragraph{STM32 NUCLEO-F401RE}\label{sec:stm32-nucleo-f401re}\mbox{}\\

\textbf{{\hyperlink{LF01_Link}{/LF01/}, \hyperlink{LF02_Link}{/LF02/}, \hyperlink{lf-audiorecord}{/LF04/}, \hyperlink{lf-audioplayback}{/LF05/}, \hyperlink{lf-pitchaudio}{/LF06/}}}

\begin{wrapfigure}{r}{0.4\textwidth} % Increase the width of the figure environment
	\vspace{-10pt}
	\hspace{20pt}
	\includegraphics[width=0.25\textwidth]{images/05_technische_spezifikation/NUCLEO-F401RE.jpg}
	\caption{STM32 NUCLEO-F401RE Board}
	\label{fig:nucleo-f401re}
\end{wrapfigure}

\vspace{1em}

Da nicht genug NUCLEO-F722 Boards zur Verfügung standen, um jede Komponente damit zu entwickeln, wurden die Audio- und Interface-Komponenten mit dem, bereits aus ES bekannten, NUCLEO-F401RE Board entwickelt. Eine Migration und anschließende Integration auf das F7 Board folgt nach der Entwicklung der Komponenten.

Die F4-CPU ist im Vergleich zur F7 Zielhardware zwar schwächer, sollte jedoch für die minimalen DSP-Operationen und Peripherien des Interface ausreichen.

Es verfügt über einen ARM Cortex-M4 Prozessor mit einer Taktfrequenz von \SI{84}{\mega\hertz}, \SI{512}{\kilo\byte} Flash-Speicher und \SI{96}{\kilo\byte} SRAM.

Das Board wird mit einer Versorgungsspannung von 5V betrieben. \cite{nucleo-f401re}

\vspace{4em}




\paragraph{STM32 Nucleo-F722ZE}\label{sec:stm32-nucleo-f722ze}\mbox{}\\

\textbf{\hyperlink{lf-nn-01}{/LF03/}} \\

\begin{wrapfigure}{r}{0.4\textwidth} % Increase the width of the figure environment
	\vspace{-10pt}
	\hspace{20pt}
	\centering
	\includegraphics[width=0.25\textwidth]{images/05_technische_spezifikation/nn/nucleo_f722ze.jpg}
	\caption{STM32 Nucleo-F722ZE Board}
	\label{fig:nucleo-f722ze}
\end{wrapfigure}


Das in \textbf{Abbildung \ref{fig:nucleo-f722ze}} gezeigte STM32 Nucleo-F722ZE Board integriert einen Microcontroller und wird in erster Linie für den Betrieb des neuronalen Netzes benötigt. Auf diesem Board soll jedoch das gesamte Projekt umgesetzt werden. Nachdem noch keine Erfahrungen mit dem Ressourcenverbrauch von neuronalen Netzen auf Microcontrollern gesammelt wurden, wurde dieses Board ausgewählt, da es über mehr Ressourcen verfügt als die Nucleo-F7401RE Boards. Dadurch soll sicher gestellt werden, dass bei der Integration aller Komponenten nicht zu Ressourcenknappheiten, insbesondere beim RAM, kommt.

Es verfügt im Vergleich zum STM32 Nucleo-F401RE Board über mehr SRAM (256 Kbytes vs. 96 Kbytes) und über eine leistungsstärkere CPU (Cortex M7 CPU mit 462 DMIPS/2.14 DMIPS vs. Cortex M4 CPU mit 105 DMIPS/1.25 DMIPS) \cite{stm32F7-board}. 

Das Board wird mit einer Versorgungsspannung von 5V betrieben. 



\newpage
\subsubsection{Schaltpläne}
\label{sec:test-schematics}

% Including the first landscape schematic PDF as a single page with label

\begin{figure}[ht]
	\centering
	\includepdf[pages=-]{../schematics/audio_test_schematic.pdf}
	\label{fig:audio_test_schematic}
\end{figure}

\newpage
\begin{figure}[ht]
	\centering
	\includepdf[pages=-]{../schematics/interface_test_schematic.pdf}
	\label{fig:interface_test_schematic}
\end{figure}
\subsubsection{LF03: Klassifizierung von Audiodateien des Benutzers durch ein neuronales Netz}
\label{sec:lf-nn-01}

\begin{table}[h!]
	\begin{tabularx}{\textwidth}{|l|X|}
		\hline
		\textbf{Priorität} & Muss \\ \hline
		\textbf{Akteur} & Benutzer \\ \hline
		\textbf{Beschreibung} & Ein neuronales Netz soll beim Einschalten des Samplers die unklassifizierten Audiosamples in für die Musikproduktion relevante Kategorien klassifizieren, wie beispielsweise Genre, Instrumententyp oder Klangfarbe.
		
		Dieser Lastenfall enthält drei Arbeitspakete:
		\begin{enumerate}
    		\item \textbf{/LF03/AP01 Auswahl geeigneter Merkmalsklassen für Audiosamples}: Diese Kategorien sollten zum einen sinnvoll bei der Musikproduktion sein, zum anderen auch mit der genutzten Hardware klassifizierbar und damit realistisch sein.
    		\item \textbf{/LF03/AP02 Entwicklung und Training des neuronalen Netzes}: Das neuronale Netz muss in der Lage sein, annährend zutreffende Klassifizierungsergebnisse auszugeben. Ziel ist ein Accuracy-Wert von über 50\%. Zum Vergleich läge der Accuracy-Wert bei \textit{$\frac{1}{5} = 20\%$}, wenn das neuronale Netz rein zufällige Ergebnisse ausgeben würde. Gleichzeitig ist dieses Ziel von 50\% realistisch erreichbar. Um dieses Ziel zu erreichen, muss das neuronale Netz ausreichend mit relevanten Daten trainiert werden. Weiterhin sollte die Confusion-Matrix eine klare Tendenz in der korrekten Klassifizierung der Kategorien aufzeigen. Darüber hinaus sollten auch manuell eingegebene Samples, wie beispielsweise ein Rock-Song, zuverlässig den entsprechenden Kategorien zugeordnet werden können. 
			\item \textbf{/LF03/AP03 Betrieb des neuronalen Netzes auf dem Nucleo-F7 Board}: Das neuronale Netz, einschließlich aller vor- und nachgelagerten Datenvorverarbeitungsschritte und der Generierung der Spektrogramme, muss auf einem Nucleo-F7 Board laufen können. 
		\end{enumerate}
		\\ \hline
	\end{tabularx}
\end{table}
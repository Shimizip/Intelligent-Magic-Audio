\subsection{PCB-Design}\label{sec:pcb-design}

Der ursprüngliche Plan, ein voll funktionales Eurorack-Modul mit eigens entwickeltem PCB, wie in \textbf{\hyperlink{lf-pcbdesign}{LF05}} ist leider aufgrund von zeitlicher Knappheit und einem zu ambitionierten Funktionsumfang aus der Projektabgabe gestrichen worden.

Hier haben schlichtweg die Vorerfahrung im PCB-Design gefehlt, um noch nebenbei ein funktionierendes Board zu entwickeln.

Jedoch ist im Nachgang begonnen worden ein solches PCB zu entwickeln, die bisherigen Schaltpläne befinden sich im Anhang \ref{sec:appendix}.

Die Levelshifting Schaltungen für die Umwandlung von Eurorack Leveln \SI{\pm 5}{\volt} zu den Pegeln des Audio Codecs \SI{\pm 0.7}{\volt_{rms}} für die Ein-/Ausgangsstufen wurden bereits entwickelt.

Die Schaltung basiert auf einem Operationsverstärker mit negativer Rückkopplung, wodurch die gewünschte Abschwächung am Eingang und Verstärkung am Ausgang erzielt wird.

Das Projekt, und somit ein eigenes PCB wird in jedem Fall Privat weiterentwickelt.
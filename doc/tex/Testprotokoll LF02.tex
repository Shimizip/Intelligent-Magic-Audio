\subsection{Testprotokoll: /LF02/}

\subsubsection{Verbindung überprüfen}
\begin{itemize}
	\item \textbf{Testfall:} Überprüfung der Verbindung zwischen Potentiometer und Mikrocontroller.
	\item \textbf{Schritte:}
	\begin{enumerate}
		\item Prüfung der Kabelverbindungen mit einem Multimeter.
		\item Ich habe sichergestellt, dass die Signale an den vorgesehenen Pins des Mikrocontrollers ankommen.
	\end{enumerate}
	\item \textbf{Erwartete Werte:}
	\begin{itemize}
		\item Kontinuität und korrekte Verbindung an allen relevanten Pins.
		\item Die gemessene Spannung betrug 3.29 V, was innerhalb des erwarteten Bereichs von 3.3 V ± 5% liegt.
	\end{itemize}
	\item \textbf{Beschreibung des Vorgehens:}
	Die Verbindungen zwischen dem Potentiometer und dem Mikrocontroller würde überprüft, indem die Kabel mit einem Multimeter auf Kontinuität getestet habe würden. Zudem wird verifiziert, dass die Signale korrekt an den vorgesehenen Pins des Mikrocontrollers ankommen.
\end{itemize}


\subsubsection{ADC-Konfiguration überprüfen}
\begin{itemize}
	\item \textbf{Testfall:} Überprüfung der ADC-Konfiguration.
	\item \textbf{Schritte:}
	\begin{enumerate}
		\item Die Auflösung, Sample-Rate und Referenzspannung des ADCs wurden überprüft.
		\item Es wurde sichergestellt, dass der ADC korrekt konfiguriert ist.
		\item Die Referenzspannung wurde mit einem Multimeter gemessen, um ihre Übereinstimmung mit den Werte im Debugger zu bestätigen.
		\item Die Sample-Rate wurde durch Analyse der ADC-Konfigurationsregister überprüft und mit den erwarteten Werten verglichen. \texttt{ADC1\_CFGR} Configuration Register.
	\end{enumerate}
	\item \textbf{Erwartete Werte:}
	\begin{itemize}
		\item Auflösung: 12-bit (0-4096 Werte).
		\item Sample-Rate entspricht den Spezifikationen.
		\item Referenzspannung entspricht der spezifizierten Spannung.
	\end{itemize}
	\item \textbf{Testergebnisse:}
	\begin{itemize}
		\item Die Auflösung des ADCs wurde korrekt auf 12-bit (0-4096 Werte) eingestellt.
		\item Die Referenzspannung wurde mit einem Multimeter gemessen und entsprach der spezifizierten Spannung.
		\item Die Sample-Rate wurde durch Überprüfung der ADC-Konfigurationsregister bestätigt und entsprach den angegebenen Spezifikationen.
		\item Die ADC-Konfiguration war ordnungsgemäß und entsprechend den Anforderungen eingerichtet.
	\end{itemize}
\end{itemize}

\textbf{Beschreibung des Vorgehens:}
Die Konfiguration des ADCs wurde durch Überprüfung der Auflösung, Sample-Rate und Referenzspannung sichergestellt. Die Referenzspannung wurde direkt mit einem Multimeter gemessen, um ihre Übereinstimmung mit den Spezifikationen zu bestätigen. Die Sample-Rate wurde durch die Analyse der ADC-Konfigurationsregister verifiziert, indem die tatsächliche Rate mit den erwarteten Werten verglichen wurde. Die Validierung erfolgte durch den Einsatz eines Debugging-Tools, um zu gewährleisten, dass der ADC gemäß den Anforderungen konfiguriert ist.


\subsubsection{DMA-Konfiguration überprüfen}
\begin{itemize}
	\item \textbf{Testfall:} Überprüfung der DMA-Konfiguration.
	\item \textbf{Schritte:}
	\begin{enumerate}
		\item Es wurde sichergestellt, dass der DMA die ADC-Daten in den Puffer \texttt{currentValues} überträgt.
		\item Die Übertragungsart und die Übertragungsrate wurden überprüft.
		\item Das Datenformat wurde auf WORD (16-Bit) konfiguriert und geprüft.
	\end{enumerate}
	\item \textbf{Erwartete Werte:}
	\begin{itemize}
		\item Korrekte Konfiguration des DMA, Übertragungsmodus auf Circular.
		\item Datenformat korrekt auf WORD (16-Bit) eingestellt.
	\end{itemize}
	\item \textbf{Testergebnisse:}
	\begin{itemize}
		\item Der DMA überträgt die ADC-Daten wie erwartet in den Puffer \texttt{currentValues}.
		\item Die Übertragungsart ist korrekt auf Circular eingestellt.
		\item Das Datenformat wurde erfolgreich auf WORD (16-Bit) konfiguriert. Die Überprüfung wurde durch Einsicht in die DMA-Register bestätigt. Insbesondere wurden die Registerwerte für `PSIZE` und `MSIZE` auf 16-Bit überprüft, um die korrekte Einstellung des Datenformats zu bestätigen.
	\end{itemize}
\end{itemize}


\textbf{Beschreibung des Vorgehens:}
Die DMA-Konfiguration wurde überprüft, indem zuerst sichergestellt wurde, dass der DMA die ADC-Daten korrekt in den Puffer \texttt{currentValues} überträgt(Debugging). Danach wurde die Übertragungsart auf Circular gesetzt und die Übertragungsrate validiert. Schließlich wurde das Datenformat auf WORD (16-Bit) konfiguriert und durch Einsicht in die entsprechenden DMA-Register überprüft, insbesondere durch Überprüfung der Registerwerte für `PSIZE` und `MSIZE`.

\subsubsection{Daten analysieren}
\begin{itemize}
	\item \textbf{Testfall:} Analyse der aufgezeichneten ADC-Werte und Prüfung der Glättung.
	\item \textbf{Schritte:}
	\begin{enumerate}
		\item Die ADC-Werte \texttt{fm.fader\_Class[]}, \texttt{adcBuffer[]}, \texttt{smoothValue[]} und \texttt{currentClassPercentADC[]} wurden mit einem Debugger analysiert.
		\item Es wurde geprüft, ob eine lineare Zunahme der Werte entsprechend der Stellung des Schiebe-Potentiometers vorliegt.
		\item Der \texttt{smoothValue[]} wurde berechnet, und es wurde auf Schwankungen und plötzliche Änderungen geprüft. Glättung erfolgte mit \texttt{100, 1000, 10000, 30000, 100000} aufeinander addierten Werten.
	\end{enumerate}
	\item \textbf{Erwartete Werte:} 
	\begin{itemize}
		\item ADC-Werte entsprechen den Positionen des Potentiometers.
		\item Die geglätteten Werte zeigen eine stabile Wertentwicklung.
		\item Keine unerwarteten Sprünge oder signifikanten Schwankungen; Grundrauschen um wenige Volt ist normal.
	\end{itemize}
	\item \textbf{Testergebnisse:}
	\begin{itemize}
		\item Die ADC-Werte \texttt{fm.fader\_Class[]}, \texttt{adcBuffer[]}, \texttt{smoothValue[]} und \texttt{currentClassPercentADC[]} wurden erfolgreich mit dem Debugger analysiert.
		\item Die Werte zeigten eine erwartungsgemäße lineare Zunahme in Abhängigkeit von der Potentiometer-Position.
		\item Der \texttt{smoothValue[]} zeigte eine stabile Wertentwicklung ohne signifikante Schwankungen oder plötzliche Änderungen.
		\item Es traten keine unerwarteten Sprünge auf. Ein sehr geringes Grundrauschen wurde erstmal als normal eingestuft. \textcolor{red}{30000} aufeinader addierte Werte dessen Mittelwert berechnet würde stellten sich jedoch am als Effzietesten raus.
	\end{itemize}
\end{itemize}

\textbf{Beschreibung des Vorgehens:}
Die ADC-Werte wurden mit einem Debugger analysiert, um sicherzustellen, dass sie der Stellung des Potentiometers entsprechen. Es wurde überprüft, ob die Werte linear ansteigen und ob der geglättete Wert \texttt{smoothValue[]} stabil bleibt. Schwankungen und plötzliche Änderungen wurden geprüft, um die Effektivität der Glättung zu bewerten. Das Grundrauschen wurde als normal eingestuft.

\textbf{Lösung um das Grundrauschen zu minimieren:}
Das Grundrauschen könnte zukünftig durch den Einsatz von Kondensatoren, z.B. 100 nF, minimiert oder beseitigt werden.
\subsubsection{Hardware}

\textcolor{red}{TODO: Hardware eklärung und Pin belegung trennen}

\textcolor{red}{TODO: PIN Belegung an mein Projekt anpassen}

\textcolor{red}{TODO: Überarbeitung der Hardware Erklärung mehr Technisch}

\textcolor{red}{TODO: Pinbelegung in einer eigene Tabelle(Mit Pull Up, Pull Down etc.)}

\textbf{Komponenten/LF1/:} \\


\textbf{(Encoder)} \\

\textbf{Elektorisch} 3.3V

\textbf{Mechanisch} GPIO Pins \textcolor{red}{PD1 PD0 PG1}
\\

Zur Auswahl der Samples und zum Navigieren durch das Menü benötigen wir einen Rotary-Encoder mit einem Switch Button.Das Empfangen der A und B Signale des Encoders erfolgt über die Pins PD0 und PD1. Der Encoder ist so konfiguriert das wen PD0 und PD1 auf high sind der Encoder den Cursor incrementiert. Wen PD0 high und PD1 low ist wird der Cursor runtergezählt.
Das Drücken des Switch über den Pin PG1. Durch das drücken des Switch-Buttons wird der Pin PG1 auf high gesetzt und die Auswahl des Samples wird übernommen.

\begin{itemize}
	\item \textbf{Präzise Steuerung:} Der Encoder ermöglicht eine präzise Steuerung des Cursors auf dem Display.
	
	\item \textbf{Benutzerfreundlichkeit:} Der Benutzer kann durch die Liste navigieren und ein Sample auswählen. Die Kombination aus Drehbewegung und Druckknopf-Funktionalität macht den Encoder intuitiv.  
\end{itemize}





\textbf{(LCD-Display)} \\

\textbf{Elektorisch} 5.0V

\textbf{Mechanisch} GPIO Pins  \textcolor{red}{PF0 PF1} \\

Zur Visualiesierung der Sampels haben wir einen Monochronen LCD-Display benutzt. Im zusammenspiel mit dem Encoder ermöglicht es eine gute Navigation durch den gewünschten Samplepool. Die Daten werden über den Output Pin  \textcolor{red}{PF0} an den SDA des Displays übertragen. Der Takt wird über den Pin  \textcolor{red}{PF1} an den SCL übertragen. Das Display wird mit 20 FPS betrieben und mit hilfe von \textcolor{red}{Timer tim5} geupdated.

\begin{itemize}
	\item \textbf{Klare Visualisierung:} LCD-Displays bieten eine klare und gut lesbare Darstellung von Text und Grafiken.
	
	\item \textbf{Energieeffizienz:} LCDs sind energieeffizient.
	
	\item \textbf{Verfügbarkeit:} LCD-Displays sind weit verbreitet und leicht zu beschaffen, was die Beschaffungskosten senkt und die Integration erleichtert.
	
	\item \textbf{Anpassbarkeit:} Sie können einfach an verschiedene Layouts und Designs angepasst werden.
\end{itemize}


Die Kombination aus Encoder und LCD-Display bietet somit eine effiziente und benutzerfreundliche Lösung für die Steuerung und Anzeige in einem Audio Sample Recorder und Playback Device.

\newpage
\textbf{Komponenten/LF2/:}\\

\textbf{(Schiebe-Potentiometer)}\\

\textbf{Elektorisch} 3.3V

\textbf{Mechanisch} GPIO Pins  \textcolor{red}{PA0 PA3 PA4 PC0 PC3} \\

Für die Filterfunktion benötigen wir 5 Potentiometer. Es wird zyklisch die Ausgangsspannung des Schleifers abgegriffen die die Teilspannung zwichen den VCC und dem GND darstellt. Dies erfolgt mit Hilfe vom ADC und dem DMA. Die Auswertung der Spannung erfolgt über die Pins  \textcolor{red}{PA0 PA3 PA4 PC0 PC3}. Die Pins  \textcolor{red}{PA0 PA3 PA4 PC0} sind für die Klassen zuständig  \textcolor{red}{PC3} für den Schwellenwert an erlaubter abweichung.

\begin{itemize}
	\item \textbf{Präzise Steuerung und feine Abstimmung:} Ein Potentiometer ermöglicht eine stufenlose und präzise Einstellung. Durch das Schieben des Potentiometers kann der Benutzer den Cursor in kleinen, genauen Schritten bewegen.
	\item \textbf{Einfache Bedienung und intuitive Nutzung:} Potentiometer sind einfach und intuitiv zu bedienen.
	\item \textbf{Robuste Mechanik und Zuverlässigkeit:} Potentiometer sind mechanisch robust und zuverlässig. Sie können eine hohe Anzahl von Schiebezyklen ohne an Genauigkeit oder Funktionalität zu verlieren.
	\item \textbf{Direkte visuelle Rückmeldung:} Durch die sofortige visuelle Rückmeldung auf dem LCD-Display kann der Benutzer sofort sehen, wie sich die Bewegung des Potentiometers auf die Position des Cursors auswirkt.
\end{itemize}

\textbf{(ADC)}\\

\textbf{ADC} (Analog Digital Converter) wird benutzt, um die Ausgangsspannung an den Schleifern zu digitalisieren und in Zahlen zu fassen. Dies erfolgt über die Pins \textcolor{red}{PA0, PA3, PA4, PC0, PC3}, die als Schnittstellen parallel der Reihenfolge der Channels \textcolor{red}{0, 2, 3, 10, 13} dienen. Jedes Mal, wenn an der Ausgangsspannung eines der Schleifer eine Veränderung wahrgenommen wird, startet eine Interrupt Service Routine, die auf einer Callback-Methode basiert. Diese berechnet dann einen geglätteten Endwert für die Ausgangsspannung. \\

\textbf{(DMA)}\\

\textbf{DMA} (Direct Memory Access) wird gestartet, um die Werte, die an den Pins des ADC liegen, zyklisch zu pollen. Er wird zeitbasiert mit einem \textcolor{red}{Timer tim5} gestartet und in der Callback nach der Glättung der kumulierten ADC-Werte gestoppt.

\textbf{\textcolor{red}{Timer:}} Timer5, Internal Clock, One Pulse Mode \\
\textbf{\textcolor{red}{Data Width:}} Word Übertragung von 32-Bit-Datenblöcken \\
\textbf{\textcolor{red}{Mode:}} Circular \\

\textbf{(LCD-Display)}\\

Der LCD-Display ist der gleiche wie im \textcolor{red}{/LF01/} beschrieben. Dieser dient zur Darstellung der Fader Einstellung in prozentualer Form.

\newpage

\subsubsection{Pinout}
\begin{longtable}[c]{|p{2.5cm}|p{1cm}|p{2.5cm}|p{2.5cm}|p{2.5cm}|p{3cm}|}
	\hline
	\textbf{Komponente} & \textbf{PIN} & \textbf{Signal-On-PIN} &  \textbf{GPIO-Mode} & \textbf{GPIO-Pull-Up/Pull-Down } & \textbf{User-Label}\\
	\hline
	Encoder Menü & PA0 & n/a & EIMRETD & PULL UP & enc\_a\_clk\_in1 \\
	\hline
	& PA1 & n/a &  INPUT & PULL UP & enc\_a\_dt\_in2 \\
	\hline
	& PA4 & n/a & EIMRETD & PULL UP & enc\_a\_switch\_in3 \\
	\hline
	ADC & PA6 & ADC1\_IN6 & ANALOG & NPU NPD & FADER1 \\
	\hline
	& PA7 & ADC1\_IN7 & ANALOG & NPU NPD & FADER2 \\
	\hline
	& PB0 & ADC1\_IN8 & ANALOG & NPU NPD & FADER3 \\
	\hline
	& PB1 & ADC1\_IN9 & ANALOG & NPU NPD & FADER4 \\
	\hline
	& PC0 & ADC1\_IN10 & ANALOG & NPU NPD & FADER5 \\	
	\hline
	I2C & PB6 &I2C1\_SCL & AFOD & PULL UP & n/a \\
	\hline
	& PB7 &I2C1\_SDA & AFOD & PULL UP & n/a \\
	\hline
	SDIO & PC8 & SDIO\_D0 & AFPP & PULL UP & n/a \\
	\hline
	& PC9 & SDIO\_D1 & AFPP & PULL UP & n/a \\
	\hline
	& PC10 & SDIO\_D2 & AFPP & PULL UP & n/a \\
	\hline
	& PC11 & SDIO\_D3 & AFPP & PULL UP & n/a \\
	\hline
	& PC12 & SDIO\_CK & AFPP & NPU NPD & n/a \\
	\hline
	& PD2 & SDIO\_CMD & AFPP & PULL UP & n/a \\
	\hline
	I2S & PB10 & IS2\_CK & AFPP  & NPU NPD & n/a \\
	\hline
	& PB12 & IS2\_WS & AFPP & NPU NPD & n/a \\
	\hline
	& PC3 & IS2\_SD & AFPP & NPU NPD &  n/a \\
	\hline
	& PC6 & IS2\_MCK & AFPP & NPU NPD &  n/a \\
	\hline
\end{longtable}

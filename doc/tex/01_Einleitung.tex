\newpage
\section{Einleitung}
<<<<<<< HEAD
\begin{itemize}
    \item Ein Projekt aus Modul ESP SS24
    \item Eigene Idee FURZ ALARM
    \item Ziel des Projekts und Motivation
    \item Grobe Projektbeschreibung (1-2 Sätze)
\end{itemize}
    
=======

Bei dem Projekt ``Intelligent Magic Audio`` (IMA) handelt es sich um ein Projekt aus dem Modul ``Embedded Systems Praktikum`` (ESP) im Sommersemester 2024 an der TH Köln.

Ziel des Projekts ist die Entwicklung eines Audiosamplers im Eurorack-Standard, der auf einem STM32 Mikrocontroller basiert. Die Besonderheit des Samplers darin, dass auf eine SD-Karte geladenene Audiosamples über ein lokal laufendes Neuronales Netz (Edge AI) in fünf Klassen klassifiziert werden. Mit Hilfe von fünf Fadern und einer Suchfunktion kann der Nutzer sich passende Samples aus seiner Audiobibliothek vorschlagen und bei Bedarf wiedergeben lassen. 

Das Projektthema wurde aus Eigeninitiative vorgeschlagen, da es zentrale Aspekte der eingebetteten Systeme wie die Signal- bzw. Audioverarbeitung und den Umgang mit Peripheriegeräten vereint. Dies ermöglicht den Teammitgliedern, ihre Kenntnisse aus dem Modul ``Embedded Systems`` (ES) zu nutzen, um ein eigens entworfenes Produkt zu entwickeln und gleichzeitig ihre Fähigkeiten in ihren persönlichen Interessensgebieten zu vertiefen. Die Integration der Edge AI-Komponente bietet zudem die Möglichkeit, Erfahrungen im Betrieb neuronaler Netze auf Mikrocontrollern zu sammeln.

Der folgende Bericht bietet eine detaillierte Übersicht über den Projektverlauf, die verwendeten Technologien, die Herausforderungen und die Ergebnisse.
>>>>>>> 181063df3dc6111efc1aea792cd1ccf600e7188a

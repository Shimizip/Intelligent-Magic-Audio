\newpage
\section{Projektanforderungen}

TODO: Nummerierung der LFs

Basierend auf dem zuvor entwickelten Produktkonzept wurden funktionale und nicht-funktionale Anforderungen abgeleitet. Sie bilden das Lastenheft des Projekts und werden in den folgenden zwei Abschnitten näher beschrieben.

\subsection{Funktionale Anforderungen}
Die funktionalen Anforderungen definieren die Funktionen, die dem Benutzer zur Verfügung gestellt werden sollen. Jede dieser Anforderungen wird mit dem Kürzel „LF“ für Lastenfall, gefolgt von einer fortlaufenden Nummer, gekennzeichnet.

\subsubsection{LF1: Steuerung des Cursors und Auswahl eines Samples}
	\hypertarget{LF01_Link}{}
\begin{longtable}[c]{|p{3cm}|p{13cm}|}
	\hline
	\textbf{Priorität} & Muss \\
	\hline
	\textbf{Akteur} & Benutzer \\
	\hline
	\textbf{Beschreibung} & 
	\begin{tabularx}{13cm}{X}
		\textbf{1. Cursor Bewegung:} \\
		\textbf{Aktion:} Der Benutzer dreht den Encoder. \\
		\textbf{Reaktion:} Der Cursor auf dem LCD-Display bewegt sich nach oben oder unten entsprechend den Stepperschritten des Encoders. \\
		\textbf{Details:} \\
		- Der Cursor bewegt sich um eine Position in der Liste pro Schritt des Encoders. \\
		- Der Cursor kann den Beginn oder das Ende der Liste erreichen, ohne die Liste über die Grenzen hinaus zu bewegen. Er spring von daher am Anfang oder Ende der Liste. \\
		\\
		\textbf{2. Sample Auswahl:} \\
		\textbf{Aktion:} Der Benutzer drückt den Switch des Encoders. \\
		\textbf{Reaktion:} Das aktuell ausgewählte Sample wird hervorgehoben und sein Name wird unter der Liste auf dem LCD-Display angezeigt. \\
		\textbf{Details:} \\
		- Der Name des ausgewählten Samples wird deutlich unter der Liste angezeigt. \\
		- Der Name sollte in einem Format angezeigt werden, das gut lesbar ist und keine Abkürzungen oder Unklarheiten aufweist. \\
	\end{tabularx} \\
	\hline
\end{longtable}

\newpage
\subsubsection{LF2: Verhalten der Fader und des Displays}
	\hypertarget{LF02_Link}{}
\begin{longtable}[c]{|p{3cm}|p{13cm}|}
	\hline
	\textbf{Priorität} & Muss \\
	\hline
	\textbf{Akteur} & Benutzer \\
	\hline
	\textbf{Beschreibung} & 
	\begin{tabularx}{13cm}{X}
		\textbf{1. Anzeige der Fader-Werte:} \\
		\textbf{Aktion:} Der Benutzer bewegt die Schiebepotentiometer. \\
		\textbf{Reaktion:} Die prozentuale Angabe der Spannung, in der sich der Potentiometer befindet, wird angezeigt. \\
		\textbf{Details:} \\
		- Die prozentualen Ausgaben spiegeln die Klassen wider, wie Rhythmic, Sustained, usw., sowie die Einstellung des Threasholds \\
		- Die Prozentsätze sollen klar und deutlich angezeigt werden, ohne Verzögerung. \\
		\\
		\textbf{2. Filtern der Samples:} \\
		\textbf{Aktion:} Der Benutzer bewegt die Schiebepotentiometer. \\
		\textbf{Reaktion:} In der Liste auf dem Display werden die Samples angezeigt dessen Audio Klassen nach der Klassiefiezierung, die den Prozentsatz der Potentiometereinstellung nicht mehr als einen bestimmten Schwellenwert überschreiten. \\
		\textbf{Details:} \\
		- Die Anzeige der Samples soll dynamisch aktualisiert werden, basierend auf den aktuellen Fader-Werten. \\
		- Der Schwellenwert kann angepasst werden, um eine präzise Filterung der Samples zu ermöglichen. \\
	\end{tabularx} \\
	\hline
\end{longtable}
\subsubsection{LF03: Klassifizierung von Audiodateien des Benutzers durch ein neuronales Netz}
\label{lf-nn-01}

\begin{table}[h!]
	\begin{tabularx}{\textwidth}{|l|X|}
		\hline
		\textbf{Priorität} & Muss \\ \hline
		\textbf{Akteur} & Benutzer \\ \hline
		\textbf{Beschreibung} & Ein neuronales Netz soll beim Einschalten des Samplers die unklassifizierten Audiosamples in für die Musikproduktion relevante Kategorien klassifizieren, wie beispielsweise Genre, Instrumententyp oder Klangfarbe.
		
		Dieser Lastenfall enthält drei Arbeitspakete:
		\begin{enumerate}
    		\item \textbf{/LF03/AP01 Auswahl geeigneter Merkmalsklassen für Audiosamples}: Diese Kategorien sollten zum einen sinnvoll bei der Musikproduktion sein, zum anderen auch mit der genutzten Hardware klassifizierbar und damit realistisch sein.
    		\item \textbf{/LF03/AP02 Entwicklung und Training des neuronalen Netzes}: Das neuronale Netz muss in der Lage sein, annährend zutreffende Klassifizierungsergebnisse auszugeben. Ziel ist ein Accuracy-Wert von über 50\%. Zum Vergleich läge der Accuracy-Wert bei \textit{$\frac{1}{5} = 20\%$}, wenn das neuronale Netz rein zufällige Ergebnisse ausgeben würde. Gleichzeitig ist dieses Ziel von 50\% realistisch erreichbar. Um dieses Ziel zu erreichen, muss das neuronale Netz ausreichend mit relevanten Daten trainiert werden. Weiterhin sollte die Confusion-Matrix eine klare Tendenz in der korrekten Klassifizierung der Kategorien aufzeigen. Darüber hinaus sollten auch manuell eingegebene Samples, wie beispielsweise ein Rock-Song, zuverlässig den entsprechenden Kategorien zugeordnet werden können. 
			\item \textbf{/LF03/AP03 Betrieb des neuronalen Netzes auf dem Nucleo-F7 Board}: Das neuronale Netz, einschließlich aller vor- und nachgelagerten Datenvorverarbeitungsschritte und der Generierung der Spektrogramme, muss auf einem Nucleo-F7 Board laufen können. 
		\end{enumerate}
		\\ \hline
	\end{tabularx}
\end{table}
\subsubsection{LF04: Aufnahme von Audioquelle über REC IN}
Das System muss eine Audioquelle, welche in der Eingangsbuchse eingesteckt ist aufnehmen können, und diese Aufnahme als korrekt kodierte PCM .wav Datei auf der SD-Karte abspeichern. Nach der Aufnahme muss eine Indexierung und Klassifizierung durch das Neuronale Netz geschehen (Wie in LFxxx beschrieben)


\newpage

\subsection{Nicht-Funktionale Anforderungen}
Die nicht-funktionalen Anforderungen spezifizieren Qualitätsmerkmale, die das System erfüllen muss, aber nicht direkt mit den spezifischen Funktionen verbunden sind. Diese Anforderungen werden mit dem Kürzel „LQ“, gefolgt von einer fortlaufenden Nummer, gekennzeichnet.


\subsubsection{Performance}

\begin{table}[h!]
	\begin{tabularx}{\textwidth}{|l|X|}
		\hline
		\textbf{/LQ1/} & \textbf{Latenz} \\ \hline
		\textbf{Kategorie} & Performance \\ \hline
		\textbf{Beschreibung} & Die Latenzzeit zwischen dem Triggern des Abspielmechanismus (Play-Button oder TRIG IN) und der Audiowiedergabe soll minimal sein (unter \SI{20}{\milli\second}), um die Funktionalität als Musikinstrument für die Live-Performance zu gewährleisten. \\ \hline
	\end{tabularx}
\end{table}

\begin{table}[h!]
	\begin{tabularx}{\textwidth}{|l|X|}
		\hline
		\textbf{\hypertarget{lq-nn-01}{/LQ02/}} & \textbf{Performante Sampleklassifizierung} \\ \hline
		\textbf{Kategorie} & Performance \\ \hline
		\textbf{Beschreibung} & Das System muss in der Lage sein, Audio-Daten schnell zu Klassifizieren, damit der Benutzer nicht in seinem kreativen Drang behindert wird. \\ \hline
	\end{tabularx}
\end{table}

\begin{table}[h!]
	\begin{tabularx}{\textwidth}{|l|X|}
		\hline
		\textbf{/LQ3/} & \textbf{Datenübertragungsrate} \\ \hline
		\textbf{Kategorie} & Performance \\ \hline
		\textbf{Beschreibung} & Die Middleware muss hohe Datenübertragungsraten unterstützen, um eine kontinuierliche und verlustfreie Audioübertragung zu gewährleisten. \\ \hline
	\end{tabularx}
\end{table}

\subsubsection{Benutzerfreundlichkeit}

\begin{table}[h!]
	\begin{tabularx}{\textwidth}{|l|X|}
		\hline
		\textbf{/LQ4/} & \textbf{Intuitive Benutzeroberfläche} \\ \hline
		\textbf{Kategorie} & Benutzerfreundlichkeit \\ \hline
		\textbf{Beschreibung} & Die Benutzeroberfläche muss intuitiv und leicht bedienbar sein, damit Nutzer direkt damit arbeiten können, ohne eine Dokumentation lesen zu müssen. \\ \hline
	\end{tabularx}
\end{table}

\newpage
\subsubsection{Effizienz}

\begin{table}[h!]
	\begin{tabularx}{\textwidth}{|l|X|}
		\hline
		\textbf{/LQ5/} & \textbf{Optimierung der Audioverarbeitung} \\ \hline
		\textbf{Kategorie} & Effizienz \\ \hline
		\textbf{Beschreibung} & Algorithmen für die Audioverarbeitung und Klassifizierung sollten so optimiert sein, dass sie minimale Rechenressourcen benötigen, ohne die Genauigkeit zu beeinträchtigen. \\ \hline
	\end{tabularx}
\end{table}
\newpage
\section{Projektanforderungen}
\begin{itemize}
    \item Lastenheft: Welche Anforderungen gibt es? Falls nicht-funktionale Anforderungen existieren: in funktionale und nicht-funktionale unterteilen; LF nummeriert.
    \item (Pflichtenheft wird in Spezifikation integriert)
\end{itemize}

\subsection{Funktionale Anforderungen}

\subsubsection{LF1: Steuerung des Cursors und Auswahl eines Samples}

\begin{longtable}[c]{|p{3cm}|p{13cm}|}
\hline
\textbf{Priorität} & Muss \\
\hline
\textbf{Akteur} & Benutzer \\
\hline
\textbf{Beschreibung} & 
\begin{tabularx}{13cm}{X}
\textbf{1. Cursor Bewegung:} \\
\textbf{Aktion:} Der Benutzer dreht den Encoder. \\
\textbf{Reaktion:} Der Cursor auf dem LCD-Display bewegt sich nach oben oder unten entsprechend den Stepperschritten des Encoders. \\
\textbf{Details:} \\
- Der Cursor bewegt sich um eine Position in der Liste pro Schritt des Encoders. \\
- Der Cursor kann den Beginn oder das Ende der Liste erreichen, ohne die Liste über die Grenzen hinaus zu bewegen. Er spring von daher am Anfang oder Ende der Liste. \\
\\
\textbf{2. Sample Auswahl:} \\
\textbf{Aktion:} Der Benutzer drückt den Switch des Encoders. \\
\textbf{Reaktion:} Das aktuell ausgewählte Sample wird hervorgehoben und sein Name wird unter der Liste auf dem LCD-Display angezeigt. \\
\textbf{Details:} \\
- Der Name des ausgewählten Samples wird deutlich unter der Liste angezeigt. \\
- Der Name sollte in einem Format angezeigt werden, das gut lesbar ist und keine Abkürzungen oder Unklarheiten aufweist. \\
\end{tabularx} \\
\hline
\end{longtable}

\newpage
\subsubsection{LF2: Verhalten der Fader und des Displays}

\begin{longtable}[c]{|p{3cm}|p{13cm}|}
\hline
\textbf{Priorität} & Muss \\
\hline
\textbf{Akteur} & Benutzer \\
\hline
\textbf{Beschreibung} & 
\begin{tabularx}{13cm}{X}
\textbf{1. Anzeige der Fader-Werte:} \\
\textbf{Aktion:} Der Benutzer bewegt die Schiebepotentiometer. \\
\textbf{Reaktion:} Die prozentuale Angabe der Spannung, in der sich der Potentiometer befindet, wird angezeigt. \\
\textbf{Details:} \\
- Die prozentualen Ausgaben spiegeln die Klassen wider, wie Rhythmic, Sustained, usw., sowie die Einstellung des Threasholds \\
- Die Prozentsätze sollen klar und deutlich angezeigt werden, ohne Verzögerung. \\
\\
\textbf{2. Filtern der Samples:} \\
\textbf{Aktion:} Der Benutzer bewegt die Schiebepotentiometer. \\
\textbf{Reaktion:} In der Liste auf dem Display werden die Samples angezeigt dessen Audio Klassen nach der Klassiefiezierung, die den Prozentsatz der Potentiometereinstellung nicht mehr als einen bestimmten Schwellenwert überschreiten. \\
\textbf{Details:} \\
- Die Anzeige der Samples soll dynamisch aktualisiert werden, basierend auf den aktuellen Fader-Werten. \\
- Der Schwellenwert kann angepasst werden, um eine präzise Filterung der Samples zu ermöglichen. \\
\end{tabularx} \\
\hline
\end{longtable}

\newpage
\subsection{Nicht-Funktionale Anforderungen}

\subsubsection{Performance}

\begin{table}[h!]
\begin{tabularx}{13cm}{|l|X|}
\hline
\textbf{/LQ1/} & \textbf{Latenz} \\ \hline
\textbf{Kategorie} & Performance \\ \hline
\textbf{Beschreibung} & Die Latenzzeit zwischen der Aufnahme eines Audio-Samples und dessen Wiedergabe sollte minimal sein (z.B. <10ms), um eine reibungslose Benutzererfahrung zu gewährleisten. \\ \hline
\end{tabularx}
\end{table}

\begin{table}[h!]
\begin{tabularx}{13cm}{|l|X|}
\hline
\textbf{/LQ2/} & \textbf{Echtzeitverarbeitung} \\ \hline
\textbf{Kategorie} & Performance \\ \hline
\textbf{Beschreibung} & Das System muss in der Lage sein, Audio-Daten in Echtzeit zu verarbeiten, um eine sofortige Rückmeldung und Sampleklassifizierung zu ermöglichen. \\ \hline
\end{tabularx}
\end{table}

\begin{table}[h!]
\begin{tabularx}{13cm}{|l|X|}
\hline
\textbf{/LQ3/} & \textbf{Datenübertragungsrate} \\ \hline
\textbf{Kategorie} & Performance \\ \hline
\textbf{Beschreibung} & Die Middleware muss hohe Datenübertragungsraten unterstützen, um eine kontinuierliche und verlustfreie Audioübertragung zu gewährleisten. \\ \hline
\end{tabularx}
\end{table}

\subsubsection{Zuverlässigkeit}

\begin{table}[h!]
\begin{tabularx}{13cm}{|l|X|}
\hline
\textbf{/LQ4/} & \textbf{Fehlertoleranz} \\ \hline
\textbf{Kategorie} & Zuverlässigkeit \\ \hline
\textbf{Beschreibung} & Das System sollte fehlertolerant sein und sich bei Hardware- oder Softwarefehlern selbstständig erholen können, um Datenverluste zu vermeiden. \\ \hline
\end{tabularx}
\end{table}

\begin{table}[h!]
\begin{tabularx}{13cm}{|l|X|}
\hline
\textbf{/LQ5/} & \textbf{Systemverfügbarkeit} \\ \hline
\textbf{Kategorie} & Zuverlässigkeit \\ \hline
\textbf{Beschreibung} & Das Gerät muss eine hohe Verfügbarkeit (z.B. 99,9\%) gewährleisten, besonders in Live-Performance-Umgebungen. \\ \hline
\end{tabularx}
\end{table}

\newpage
\subsubsection{Benutzerfreundlichkeit}

\begin{table}[h!]
\begin{tabularx}{13cm}{|l|X|}
\hline
\textbf{/LQ6/} & \textbf{Intuitive Benutzeroberfläche} \\ \hline
\textbf{Kategorie} & Benutzerfreundlichkeit \\ \hline
\textbf{Beschreibung} & Die UI muss intuitiv und leicht bedienbar sein, damit Nutzer ohne umfangreiche Schulung schnell damit arbeiten können. \\ \hline
\end{tabularx}
\end{table}

\begin{table}[h!]
\begin{tabularx}{13cm}{|l|X|}
\hline
\textbf{/LQ7/} & \textbf{Feedback-Mechanismen} \\ \hline
\textbf{Kategorie} & Benutzerfreundlichkeit \\ \hline
\textbf{Beschreibung} & Das System sollte klares und sofortiges Feedback auf Benutzeraktionen geben, um Fehlbedienungen zu minimieren. \\ \hline
\end{tabularx}
\end{table}

\subsubsection{Sicherheit}

\begin{table}[h!]
\begin{tabularx}{13cm}{|l|X|}
\hline
\textbf{/LQ8/} & \textbf{Datenintegrität} \\ \hline
\textbf{Kategorie} & Sicherheit \\ \hline
\textbf{Beschreibung} & Die Integrität der Audio-Daten muss zu jeder Zeit gewährleistet sein, insbesondere bei der Speicherung und Übertragung. \\ \hline
\end{tabularx}
\end{table}

\begin{table}[h!]
\begin{tabularx}{13cm}{|l|X|}
\hline
\textbf{/LQ9/} & \textbf{Zugriffskontrollen} \\ \hline
\textbf{Kategorie} & Sicherheit \\ \hline
\textbf{Beschreibung} & Nur autorisierte Benutzer sollten auf bestimmte Funktionen und Daten zugreifen können. \\ \hline
\end{tabularx}
\end{table}

\subsubsection{Wartbarkeit}

\begin{table}[h!]
\begin{tabularx}{13cm}{|l|X|}
\hline
\textbf{/LQ10/} & \textbf{Modularität} \\ \hline
\textbf{Kategorie} & Wartbarkeit \\ \hline
\textbf{Beschreibung} & Die Software- und Hardwarekomponenten sollten modular aufgebaut sein, um einfache Wartung und Updates zu ermöglichen. \\ \hline
\end{tabularx}
\end{table}

\begin{table}[h!]
\begin{tabularx}{13cm}{|l|X|}
\hline
\textbf{/LQ11/} & \textbf{Dokumentation} \\ \hline
\textbf{Kategorie} & Wartbarkeit \\ \hline
\textbf{Beschreibung} & Umfassende und verständliche Dokumentation für Entwickler und Benutzer sollte vorhanden sein, um Wartung und Weiterentwicklung zu erleichtern. \\ \hline
\end{tabularx}
\end{table}

\subsubsection{Portabilität}

\begin{table}[h!]
\begin{tabularx}{13cm}{|l|X|}
\hline
\textbf{/LQ12/} & \textbf{Plattformunabhängigkeit} \\ \hline
\textbf{Kategorie} & Portabilität \\ \hline
\textbf{Beschreibung} & Die Middleware sollte auf verschiedenen Hardwareplattformen laufen können, um die Flexibilität und Wiederverwendbarkeit zu erhöhen. \\ \hline
\end{tabularx}
\end{table}

\begin{table}[h!]
\begin{tabularx}{13cm}{|l|X|}
\hline
\textbf{/LQ13/} & \textbf{Einfache Installation} \\ \hline
\textbf{Kategorie} & Portabilität \\ \hline
\textbf{Beschreibung} & Das System sollte einfach zu installieren und zu konfigurieren sein, um den Einsatz in verschiedenen Umgebungen zu erleichtern. \\ \hline
\end{tabularx}
\end{table}

\subsubsection{Effizienz}

\begin{table}[h!]
\begin{tabularx}{13cm}{|l|X|}
\hline
\textbf{/LQ14/} & \textbf{Ressourcennutzung} \\ \hline
\textbf{Kategorie} & Effizienz \\ \hline
\textbf{Beschreibung} & Das System sollte sparsam mit den verfügbaren Ressourcen (CPU, Speicher, Energie) umgehen, um eine lange Betriebsdauer und Stabilität zu gewährleisten. \\ \hline
\end{tabularx}
\end{table}

\begin{table}[h!]
\begin{tabularx}{13cm}{|l|X|}
\hline
\textbf{/LQ15/} & \textbf{Optimierung der Audioverarbeitung} \\ \hline
\textbf{Kategorie} & Effizienz \\ \hline
\textbf{Beschreibung} & Algorithmen für die Audioverarbeitung und Klassifizierung sollten so optimiert sein, dass sie minimale Rechenressourcen benötigen, ohne die Genauigkeit zu beeinträchtigen. \\ \hline
\end{tabularx}
\end{table}

\subsubsection{Kompatibilität}

\begin{table}[h!]
\begin{tabularx}{13cm}{|l|X|}
\hline
\textbf{/LQ16/} & \textbf{Schnittstellen} \\ \hline
\textbf{Kategorie} & Kompatibilität \\ \hline
\textbf{Beschreibung} & Das System sollte mit gängigen Schnittstellen und Protokollen kompatibel sein, um eine einfache Integration in bestehende Setups zu ermöglichen. \\ \hline
\end{tabularx}
\end{table}

\begin{table}[h!]
\begin{tabularx}{13cm}{|l|X|}
\hline
\textbf{/LQ17/} & \textbf{Erweiterbarkeit} \\ \hline
\textbf{Kategorie} & Kompatibilität \\ \hline
\textbf{Beschreibung} & Das Gerät sollte leicht erweiterbar sein, um zukünftige Funktionalitäten und Verbesserungen zu unterstützen. \\ \hline
\end{tabularx}
\end{table}

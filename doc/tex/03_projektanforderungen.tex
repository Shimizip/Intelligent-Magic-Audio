\newpage
\section{Projektanforderungen}

Basierend auf dem zuvor entwickelten Produktkonzept wurden funktionale und nicht-funktionale Anforderungen abgeleitet. Sie bilden das Lastenheft des Projekts und werden in den folgenden zwei Abschnitten näher beschrieben.

\subsection{Funktionale Anforderungen}
Die funktionalen Anforderungen definieren die Funktionen, die dem Benutzer zur Verfügung gestellt werden sollen. Jede dieser Anforderungen wird mit dem Kürzel „LF“ für Lastenfall, gefolgt von einer fortlaufenden Nummer, gekennzeichnet.

\include{Interface/LF Interface}
\subsubsection{LF03: Klassifizierung von Audiodateien des Benutzers durch ein neuronales Netz}
\label{lf-nn-01}

\begin{table}[h!]
	\begin{tabularx}{\textwidth}{|l|X|}
		\hline
		\textbf{Priorität} & Muss \\ \hline
		\textbf{Akteur} & Benutzer \\ \hline
		\textbf{Beschreibung} & Ein neuronales Netz soll beim Einschalten des Samplers die unklassifizierten Audiosamples in für die Musikproduktion relevante Kategorien klassifizieren, wie beispielsweise Genre, Instrumententyp oder Klangfarbe.
		
		Dieser Lastenfall enthält drei Arbeitspakete:
		\begin{enumerate}
    		\item \textbf{/LF03/AP01 Auswahl geeigneter Merkmalsklassen für Audiosamples}: Diese Kategorien sollten zum einen sinnvoll bei der Musikproduktion sein, zum anderen auch mit der genutzten Hardware klassifizierbar und damit realistisch sein.
    		\item \textbf{/LF03/AP02 Entwicklung und Training des neuronalen Netzes}: Das neuronale Netz muss in der Lage sein, annährend zutreffende Klassifizierungsergebnisse auszugeben. Ziel ist ein Accuracy-Wert von über 50\%. Zum Vergleich läge der Accuracy-Wert bei \textit{$\frac{1}{5} = 20\%$}, wenn das neuronale Netz rein zufällige Ergebnisse ausgeben würde. Gleichzeitig ist dieses Ziel von 50\% realistisch erreichbar. Um dieses Ziel zu erreichen, muss das neuronale Netz ausreichend mit relevanten Daten trainiert werden. Weiterhin sollte die Confusion-Matrix eine klare Tendenz in der korrekten Klassifizierung der Kategorien aufzeigen. Darüber hinaus sollten auch manuell eingegebene Samples, wie beispielsweise ein Rock-Song, zuverlässig den entsprechenden Kategorien zugeordnet werden können. 
			\item \textbf{/LF03/AP03 Betrieb des neuronalen Netzes auf dem Nucleo-F7 Board}: Das neuronale Netz, einschließlich aller vor- und nachgelagerten Datenvorverarbeitungsschritte und der Generierung der Spektrogramme, muss auf einem Nucleo-F7 Board laufen können. 
		\end{enumerate}
		\\ \hline
	\end{tabularx}
\end{table}

\subsubsection{LF04: Aufnahme von Audioquelle über REC IN}
\hypertarget{lf-audiorecord}{}

\begin{table}[h!]
	\begin{tabularx}{\textwidth}{|l|X|}
		\hline
		\textbf{Priorität} & Muss \\ \hline
		\textbf{Akteur} & Benutzer \\ \hline
		\textbf{Beschreibung} & Das System muss eine Audioquelle mit Line-Pegel, welche in der Eingangsbuchse eingesteckt ist aufnehmen können, und diese Aufnahme als korrekt kodierte PCM .wav Datei auf der SD-Karte abspeichern. Nach der Aufnahme muss eine Indexierung und Klassifizierung durch das Neuronale Netz geschehen (Wie in \textbf{\hyperlink{lf-nn-01}{LF03}} beschrieben)
		\\ \hline
	\end{tabularx}
\end{table}




\subsubsection{LF05: Wiedergabe von Audiodaten über OUT}
\hypertarget{lf-audioplayback}{}

\begin{table}[h!]
	\begin{tabularx}{\textwidth}{|l|X|}
		\hline
		\textbf{Priorität} & Muss \\ \hline
		\textbf{Akteur} & Benutzer \\ \hline
		\textbf{Beschreibung} & Das System muss Stereo PCM .wav Audiodaten, welche auf der SD-Karte gespeichert sind, über den Audiocodec abspielen können.
		
		Ziel ist eine zuverlässige, saubere Audiowiedergabe, die keine Störgeräusche wie Knackser und andere Artifakte produziert.
		\\ \hline
	\end{tabularx}
\end{table}



\subsubsection{LF06: Dynamische Änderung der Tonhöhe/Abspielgeschwindigkeit}
\hypertarget{lf-pitchaudio}{}

\begin{table}[h!]
	\begin{tabularx}{\textwidth}{|l|X|}
		\hline
		\textbf{Priorität} & Muss \\ \hline
		\textbf{Akteur} & Benutzer \\ \hline
		\textbf{Beschreibung} & Das System muss die Audiodaten (wie in \textbf{\hyperlink{lf-audioplayback}{LF05}} beschrieben) in verschiedenen Abspielgeschwindigkeiten wiedergeben können. So kann die Tonhöhe angepasst werden.
		
		Hierbei ist ein Algorithmus zu implementieren, der eine auditiv qualitative Tonhöhenanpassung umsetzt. Artifakte sollen vermieden werden.
		\\ \hline
	\end{tabularx}
\end{table}



\newpage

\subsection{Nicht-Funktionale Anforderungen}
Die nicht-funktionalen Anforderungen spezifizieren Qualitätsmerkmale, die das System erfüllen muss, aber nicht direkt mit den spezifischen Funktionen verbunden sind. Diese Anforderungen werden mit dem Kürzel „LQ“, gefolgt von einer fortlaufenden Nummer, gekennzeichnet.


\subsubsection{Performance}

\begin{table}[h!]
	\begin{tabularx}{\textwidth}{|l|X|}
		\hline
		\textbf{/LQ1/} & \textbf{Latenz} \\ \hline
		\textbf{Kategorie} & Performance \\ \hline
		\textbf{Beschreibung} & Die Latenzzeit zwischen der Aufnahme eines Audio-Samples und dessen Wiedergabe sollte minimal sein (z.B. <10ms), um eine reibungslose Benutzererfahrung zu gewährleisten. \\ \hline
	\end{tabularx}
\end{table}

\begin{table}[h!]
	\begin{tabularx}{\textwidth}{|l|X|}
		\hline
		\textbf{/LQ2/} & \textbf{Echtzeitverarbeitung} \\ \hline
		\textbf{Kategorie} & Performance \\ \hline
		\textbf{Beschreibung} & Das System muss in der Lage sein, Audio-Daten in Echtzeit zu verarbeiten, um eine sofortige Rückmeldung und Sampleklassifizierung zu ermöglichen. \\ \hline
	\end{tabularx}
\end{table}

\begin{table}[h!]
	\begin{tabularx}{\textwidth}{|l|X|}
		\hline
		\textbf{/LQ3/} & \textbf{Datenübertragungsrate} \\ \hline
		\textbf{Kategorie} & Performance \\ \hline
		\textbf{Beschreibung} & Die Middleware muss hohe Datenübertragungsraten unterstützen, um eine kontinuierliche und verlustfreie Audioübertragung zu gewährleisten. \\ \hline
	\end{tabularx}
\end{table}

\subsubsection{Benutzerfreundlichkeit}

\begin{table}[h!]
	\begin{tabularx}{\textwidth}{|l|X|}
		\hline
		\textbf{/LQ4/} & \textbf{Intuitive Benutzeroberfläche} \\ \hline
		\textbf{Kategorie} & Benutzerfreundlichkeit \\ \hline
		\textbf{Beschreibung} & Die Benutzeroberfläche muss intuitiv und leicht bedienbar sein, damit Nutzer ohne umfangreiche Schulung schnell damit arbeiten können. \\ \hline
	\end{tabularx}
\end{table}

\begin{table}[h!]
	\begin{tabularx}{\textwidth}{|l|X|}
		\hline
		\textbf{/LQ5/} & \textbf{Feedback-Mechanismen} \\ \hline
		\textbf{Kategorie} & Benutzerfreundlichkeit \\ \hline
		\textbf{Beschreibung} & Das System sollte klares und sofortiges Feedback auf Benutzeraktionen geben, um Fehlbedienungen zu minimieren. \\ \hline
	\end{tabularx}
\end{table}

\newpage
\subsubsection{Effizienz}

\begin{table}[h!]
	\begin{tabularx}{\textwidth}{|l|X|}
		\hline
		\textbf{/LQ6/} & \textbf{Ressourcennutzung} \\ \hline
		\textbf{Kategorie} & Effizienz \\ \hline
		\textbf{Beschreibung} & Das System sollte sparsam mit den verfügbaren Ressourcen (CPU, Speicher, Energie) umgehen, um eine lange Betriebsdauer und Stabilität zu gewährleisten. \\ \hline
	\end{tabularx}
\end{table}

\begin{table}[h!]
	\begin{tabularx}{\textwidth}{|l|X|}
		\hline
		\textbf{/LQ7/} & \textbf{Optimierung der Audioverarbeitung} \\ \hline
		\textbf{Kategorie} & Effizienz \\ \hline
		\textbf{Beschreibung} & Algorithmen für die Audioverarbeitung und Klassifizierung sollten so optimiert sein, dass sie minimale Rechenressourcen benötigen, ohne die Genauigkeit zu beeinträchtigen. \\ \hline
	\end{tabularx}
\end{table}
\subsection{Testprotokoll: /LF01/}
\textbf{\hyperlink{LF01_Link}{LF01}} \\

\subsubsection{Auswertung der Drehung des Encoders}
\begin{itemize}
	\item \textbf{Testfall:} Prüfung des Drehverhaltens
	\item \textbf{Schritte:}
	\begin{enumerate}
		\item Der Encoder wird schnell und langsam in beide Richtungen gedreht.
		\item Dabei wird die Zählvariable \texttt{rotary\_enc\_count} beobachtet.
	\end{enumerate}
	\item \textbf{Erwartete Werte:}
	\begin{itemize}
		\item Die Zählvariable zählt wie erwartet hoch und runter.
		\item Nur beim Betreten der \texttt{HAL\_GPIO\_EXTI\_Callback} soll die Variable hochgezählt werden.
	\end{itemize}
	\begin{itemize}
		\item Es traten keine Mehrfachauslösungen auf.
		\item Die Zählvariable zählte korrekt und reagierte wie erwartet auf schnelle und langsame Drehungen in beide Richtungen.
	\end{itemize}
\end{itemize}




\textbf{Beschreibung des Vorgehens:}
Um das Drehverhalten des Encoders zu testen, würde de Encoder an das System angeschlossen und die Funktionalität überprüft. Anschließend haben wir den Encoder in beiden Richtungen, sowohl schnell als auch langsam, gedreht und die Zählvariable 
Die Zählvariable \texttt{rotary\_enc\_count} stand dabei unter Beobachtung, um sicherzustellen, dass sie ausschließlich durch die \texttt{HAL\_GPIO\_EXTI\_Callback} beeinflusst wird.


\subsubsection{Debouncing des Push-Buttons}
\begin{itemize}
	\item \textbf{Testfall:} Prüfung der Funktionalität des Push-Buttons.
	\item \textbf{Schritte:}
	\begin{enumerate}
		\item Erstellung einer Zähl Variable \boldinline{cnt}
		\item Drücken des Push-Button.
		\item Prüfen ob der Interrupt korrekt ausgelöst wird.
		\item Das korrekte Verhalten des Entprellen prüfen.
	\end{enumerate}
	\item \textbf{Erwartete Werte:}
	\begin{itemize}
		\item Der Interrupt wird bei jedem Drücken sauber und eindeutig ausgelöst.
		\item Es treten kein Prellen oder Mehrfachauslösungen auf.
	\end{itemize}
	\item \textbf{Testergebnisse:}
	\begin{itemize}
		\item Der Interrupt wurde bei jedem Drücken zuverlässig und eindeutig ausgelöst.
		\item Es traten keine Prellen oder Mehrfachauslösungen auf. Die \boldinline{cnt} hat korrekt hochgezählt.
	\end{itemize}
\end{itemize}


\textbf{Beschreibung des Vorgehens:}
Der Push-Button würde mehrmals gedrückt und dabei beobachtet, ob der Interrupt korrekt ausgelöst wurde. Während des Tests wird darauf geachtet, ob der Button prellte oder Mehrfachauslösungen verursachte in dem ein Zähl Variable \texttt{cnt} im Debugger beobachte würde. Die Ergebnisse bestätigten, dass der Interrupt zuverlässig und ohne Prellen funktionierte.

\subsubsection{Menünavigation}
\begin{itemize}
	\item \textbf{Testfall:} Testen der Navigation im Menü mithilfe des Encoders.
	\item \textbf{Schritte:}
	\begin{enumerate}
		\item Den Encoder wird in beide Richtungen gedreht und die Cursor-Bewegung beobachtet. Die cursor Variable \texttt{cursor\_index} des Filemanagers  im Debugger prüfen auf deren Zählverhalten.
		\item Der Push-Button wird gedrückt, um eine Auswahl zu treffen. Der Flag switch\_push\_button prüfen um die Validierung des korrekten Verhaltens beim drücken zu bestätigen.
	\end{enumerate}
	\item \textbf{Erwartete Werte:}
	\begin{itemize}
		\item Der Cursor bewegt sich entsprechend der Drehrichtung des Encoders.
		\item Die Auswahl wird korrekt angezeigt, wenn der Push-Button gedrückt wird.
	\end{itemize}
	
	\item \textbf{Testergebnisse:}
	\begin{itemize}
		\item Der Cursor bewegte sich korrekt und präzise entsprechend der Encoder-Drehung.
		\item Die Auswahl wurde zuverlässig angezeigt, nachdem der Push-Button gedrückt wurde.
	\end{itemize}
\end{itemize}

\textbf{Beschreibung des Vorgehens:} 
Der Encoder wird gedreht, um sicherzustellen, dass der Cursor im Menü korrekt bewegt wird. Die Cursor Variable \texttt{cursor\_index} im FileManager Struct \texttt{fm} wird korrekt hoch und runter gezählt bei entsprechender Bewegung. Anschließend habe ich den Push-Button betätigt, um zu prüfen, ob die Auswahl entsprechend angezeigt wird.  Der Flag \texttt{switch\_push\_button} würde ebenfalls korrekt gesetzt.
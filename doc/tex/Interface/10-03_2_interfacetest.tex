\subsection{Testprotokoll: /LF02/}
\textbf{\hyperlink{LF02_Link}{LF02}} \\

\subsubsection{ADC-Konfiguration überprüfen}
\begin{itemize}
	\item \textbf{Testfall:} Überprüfung der ADC-Konfiguration.
	\item \textbf{Schritte:}
	\begin{enumerate}
		\item Die Auflösung, Sample-Rate und Referenzspannung des ADCs wurden überprüft.
		\item Es wird sichergestellt, dass der ADC korrekt konfiguriert ist.
		\item Die Referenzspannung wird mit einem Multimeter gemessen, um ihre Übereinstimmung mit den Werte im Debugger zu bestätigen.
		\item Die Sample-Rate wird durch Analyse der ADC-Konfigurationsregister überprüft und mit den erwarteten Werten verglichen. \texttt{ADC1\_CFGR} Configuration Register.
	\end{enumerate}
	\item \textbf{Erwartete Werte:}
	\begin{itemize}
		\item Auflösung: 12-bit (0-4096 Werte).
		\item Sample-Rate entspricht den Spezifikationen.
		\item Referenzspannung entspricht der spezifizierten Spannung.
	\end{itemize}
	\item \textbf{Testergebnisse:}
	\begin{itemize}
		\item Die Auflösung des ADCs wurde korrekt auf 12-bit (0-4096 Werte) eingestellt.
		\item Die Referenzspannung wurde mit einem Multimeter gemessen und entsprach der spezifizierten Spannung.
		\item Die Sample-Rate wurde durch Überprüfung der ADC-Konfigurationsregister bestätigt und entsprach den angegebenen Spezifikationen.
		\item Die ADC-Konfiguration war ordnungsgemäß und entsprechend den Anforderungen eingerichtet.
	\end{itemize}
\end{itemize}

\textbf{Beschreibung des Vorgehens:}
Die Konfiguration des ADCs wurde durch Überprüfung der Auflösung, Sample-Rate und Referenzspannung sichergestellt. Die Referenzspannung wurde direkt mit einem Multimeter gemessen, um ihre Übereinstimmung mit den Spezifikationen zu bestätigen. Die Sample-Rate wurde durch die Analyse der ADC-Konfigurationsregister verifiziert, indem die tatsächliche Rate mit den erwarteten Werten verglichen wurde. Die Validierung erfolgte durch den Einsatz eines Debugging-Tools, um zu gewährleisten, dass der ADC gemäß den Anforderungen konfiguriert ist.


\subsubsection{DMA-Konfiguration überprüfen}
\begin{itemize}
	\item \textbf{Testfall:} Überprüfung der DMA-Konfiguration.
	\item \textbf{Schritte:}
	\begin{enumerate}
		\item Es wird sichergestellt, dass der DMA die ADC-Daten in den Puffer \texttt{currentValues} überträgt.
		\item Die Übertragungsart und die Übertragungsrate werden überprüft.
		\item Das Datenformat wird auf WORD (16-Bit) konfiguriert und geprüft.
	\end{enumerate}
	\item \textbf{Erwartete Werte:}
	\begin{itemize}
		\item Korrekte Konfiguration des DMA, Übertragungsmodus auf Circular.
		\item Datenformat korrekt auf WORD (16-Bit) eingestellt.
	\end{itemize}
	\item \textbf{Testergebnisse:}
	\begin{itemize}
		\item Der DMA überträgt die ADC-Daten wie erwartet in den Puffer \texttt{currentValues}.
		\item Die Übertragungsart ist korrekt auf Circular eingestellt.
		\item Das Datenformat wurde erfolgreich auf WORD (16-Bit) konfiguriert. Die Überprüfung wurde durch Einsicht in die DMA-Register bestätigt. Insbesondere wurden die Registerwerte für `PSIZE` und `MSIZE` auf 16-Bit überprüft, um die korrekte Einstellung des Datenformats zu bestätigen.
	\end{itemize}
\end{itemize}


\textbf{Beschreibung des Vorgehens:}
Die DMA-Konfiguration wurde überprüft, indem zuerst sichergestellt wurde, dass der DMA die ADC-Daten korrekt in den Puffer \texttt{currentValues} überträgt(Debugging). Danach wurde die Übertragungsart auf Circular gesetzt und die Übertragungsrate validiert. Schließlich wurde das Datenformat auf WORD (16-Bit) konfiguriert und durch Einsicht in die entsprechenden DMA-Register überprüft, insbesondere durch Überprüfung der Registerwerte für `PSIZE` und `MSIZE`.

\subsubsection{Daten analysieren}
\begin{itemize}
	\item \textbf{Testfall:} Analyse der aufgezeichneten ADC-Werte und Prüfung der Glättung.
	\item \textbf{Schritte:}
	\begin{enumerate}
		\item Die ADC-Werte \texttt{fm.fader\_Class[]}, \texttt{adcBuffer[]}, \texttt{smoothValue[]} und \texttt{currentClassPercentADC[]} werden mit einem Debugger analysiert.
		\item Es wird geprüft, ob eine lineare Zunahme der Werte entsprechend der Stellung des Schiebe-Potentiometers vorliegt.
		\item Der \texttt{smoothValue[]} wird berechnet, und es wird auf Schwankungen und plötzliche Änderungen geprüft. Glättung erfolgt mit \texttt{100, 1000, 10000, 30000, 100000} aufeinander addierten Werten.
	\end{enumerate}
	\item \textbf{Erwartete Werte:} 
	\begin{itemize}
		\item ADC-Werte entsprechen den Positionen des Potentiometers.
		\item Die geglätteten Werte zeigen eine stabile Wertentwicklung.
		\item Keine unerwarteten Sprünge oder signifikanten Schwankungen; Grundrauschen um wenige Volt ist normal.
	\end{itemize}
	\item \textbf{Testergebnisse:}
	\begin{itemize}
		\item Die ADC-Werte \texttt{fm.fader\_Class[]}, \texttt{adcBuffer[]}, \texttt{smoothValue[]} und \texttt{currentClassPercentADC[]} wurden erfolgreich mit dem Debugger analysiert.
		\item Die Werte zeigten eine erwartungsgemäße lineare Zunahme in Abhängigkeit von der Potentiometer-Position.
		\item Der \texttt{smoothValue[]} zeigte eine stabile Wertentwicklung ohne signifikante Schwankungen oder plötzliche Änderungen.
		\item Es traten keine unerwarteten Sprünge auf. Ein sehr geringes Grundrauschen wurde erstmal als normal eingestuft.\textbf{30000} aufeinader addierte Werte dessen Mittelwert berechnet würde stellten sich jedoch am als Effzietesten raus.
	\end{itemize}
\end{itemize}

\textbf{Beschreibung des Vorgehens:}
Die ADC-Werte wurden mit einem Debugger analysiert, um sicherzustellen, dass sie der Stellung des Potentiometers entsprechen. Es wurde überprüft, ob die Werte linear ansteigen und ob der geglättete Wert \texttt{smoothValue[]} stabil bleibt. Schwankungen und plötzliche Änderungen wurden geprüft, um die Effektivität der Glättung zu bewerten. Das Grundrauschen wurde als normal eingestuft.

\textbf{Lösung um das Grundrauschen zu minimieren:}
Das Grundrauschen könnte zukünftig durch den Einsatz von Kondensatoren, z.B. 100 nF, minimiert oder beseitigt werden.

\newpage
\subsubsection{Wiedergabe der Fader-Werte und Filterung auf dem Display}

\begin{itemize}
	\item \textbf{Testfall:} Überprüfung der korrekten Anzeige der Fader-Werte und der Filterung der Samples auf dem Display.
	\item \textbf{Schritte:}
	\begin{enumerate}
		\item Fader-Werte anzeigen:
		\begin{enumerate}
			\item Bewege die Schiebepotentiometer (Fader) zyklisch.
			\item Überprüfe, ob die prozentualen Werte der Potentiometer korrekt und in Echtzeit auf dem Display angezeigt werden.
			\item Vergewissere dich, dass die angezeigten Werte die richtige Spannung und die korrekte Zuordnung zu den Klassen (wie Rhythmic, Sustained, usw.) widerspiegeln.
		\end{enumerate}
		\item Filterung der Samples:
		\begin{enumerate}
			\item Bewege die Schiebepotentiometer und stelle den Schwellenwert für die Filterung ein.
			\item Überprüfe, ob die Liste der Samples auf dem Display dynamisch aktualisiert wird und nur die Samples anzeigt, deren Audio-Klassen nach der Klassifizierung den Schwellenwert nicht überschreiten.
			\item Teste verschiedene Schwellenwerte und überprüfe, ob die angezeigten Samples korrekt gefiltert werden.
		\end{enumerate}
	\end{enumerate}
	\item \textbf{Erwartete Ergebnisse:}
	\begin{itemize}
		\item Die prozentualen Werte der Fader werden korrekt und in Echtzeit auf dem Display angezeigt.
		\item Die Klassen der Potentiometer-Einstellungen werden korrekt auf dem Display angezeigt.
		\item Die Liste der Samples wird dynamisch und korrekt basierend auf den Potentiometer-Werten und dem Schwellenwert aktualisiert.
		\item Keine Anzeigeprobleme oder Verzögerungen bei der Aktualisierung des Displays.
	\end{itemize}
	\item \textbf{Testergebnisse:}
	\begin{itemize}
		\item Die prozentualen Fader-Werte wurden korrekt auf dem Display angezeigt und spiegelten die realen Werte der Potentiometer wider.
		\item Die Filterung der Samples funktionierte wie erwartet; nur die Samples, die den eingestellten Schwellenwert erfüllten, wurden angezeigt.
		\item Die Anzeige auf dem Display war frei von Verzögerungen oder Fehlern.
	\end{itemize}
\end{itemize}


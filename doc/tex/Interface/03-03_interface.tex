\subsubsection{LF1: Steuerung des Cursors und Auswahl eines Samples}
	\hypertarget{LF01_Link}{}
\begin{longtable}[c]{|p{3cm}|p{13cm}|}
	\hline
	\textbf{Priorität} & Muss \\
	\hline
	\textbf{Akteur} & Benutzer \\
	\hline
	\textbf{Beschreibung} & 
	\begin{tabularx}{13cm}{X}
		\textbf{1. Cursor Bewegung:} \\
		\textbf{Aktion:} Der Benutzer dreht den Encoder. \\
		\textbf{Reaktion:} Der Cursor auf dem LCD-Display bewegt sich nach oben oder unten entsprechend den Stepperschritten des Encoders. \\
		\textbf{Details:} \\
		- Der Cursor bewegt sich um eine Position in der Liste pro Schritt des Encoders. \\
		- Der Cursor kann den Beginn oder das Ende der Liste erreichen, ohne die Liste über die Grenzen hinaus zu bewegen. Er spring von daher am Anfang oder Ende der Liste. \\
		\\
		\textbf{2. Sample Auswahl:} \\
		\textbf{Aktion:} Der Benutzer drückt den Switch des Encoders. \\
		\textbf{Reaktion:} Das aktuell ausgewählte Sample wird hervorgehoben und sein Name wird unter der Liste auf dem LCD-Display angezeigt. \\
		\textbf{Details:} \\
		- Der Name des ausgewählten Samples wird deutlich unter der Liste angezeigt. \\
		- Der Name sollte in einem Format angezeigt werden, das gut lesbar ist und keine Abkürzungen oder Unklarheiten aufweist. \\
	\end{tabularx} \\
	\hline
\end{longtable}

\newpage
\subsubsection{LF2: Verhalten der Fader und des Displays}
	\hypertarget{LF02_Link}{}
\begin{longtable}[c]{|p{3cm}|p{13cm}|}
	\hline
	\textbf{Priorität} & Muss \\
	\hline
	\textbf{Akteur} & Benutzer \\
	\hline
	\textbf{Beschreibung} & 
	\begin{tabularx}{13cm}{X}
		\textbf{1. Anzeige der Fader-Werte:} \\
		\textbf{Aktion:} Der Benutzer bewegt die Schiebepotentiometer. \\
		\textbf{Reaktion:} Die prozentuale Angabe der Spannung, in der sich der Potentiometer befindet, wird angezeigt. \\
		\textbf{Details:} \\
		- Die prozentualen Ausgaben spiegeln die Klassen wider, wie Rhythmic, Sustained, usw., sowie die Einstellung des Threasholds \\
		- Die Prozentsätze sollen klar und deutlich angezeigt werden, ohne Verzögerung. \\
		\\
		\textbf{2. Filtern der Samples:} \\
		\textbf{Aktion:} Der Benutzer bewegt die Schiebepotentiometer. \\
		\textbf{Reaktion:} In der Liste auf dem Display werden die Samples angezeigt dessen Audio Klassen nach der Klassiefiezierung, die den Prozentsatz der Potentiometereinstellung nicht mehr als einen bestimmten Schwellenwert überschreiten. \\
		\textbf{Details:} \\
		- Die Anzeige der Samples soll dynamisch aktualisiert werden, basierend auf den aktuellen Fader-Werten. \\
		- Der Schwellenwert kann angepasst werden, um eine präzise Filterung der Samples zu ermöglichen. \\
	\end{tabularx} \\
	\hline
\end{longtable}
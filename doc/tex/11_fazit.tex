\newpage
\section{Fazit}

Im Rahmen des Projekts konnten die in der Vorlesung ``Embedded Systems`` (ES) erlernten Kenntnisse weiter vertieft und um einige Themenbereiche, Technologien und Protokolle erweitert werden. 
Die praktische Anwendung dieser Kenntnisse ermöglichte es den Gruppenmitgliedern, spezialisiertes Wissen und wertvolle Erfahrungen zu sammeln, die sowohl technische als auch organisatorische Fähigkeiten umfassten. Die eigenständige Entwicklung eines Systems von der Idee bis zur Implementierung förderte nicht nur das tiefgreifende Verständnis für die verwendeten Technologien, sondern auch die Teamfähigkeit und Problemlösungskompetenz der Teilnehmer.

Die Herausforderungen bei der Umsetzung komplexer Funktionalitäten und die damit verbundenen Probleme führten jedoch zu Zeitverzögerungen, die eine vollständige Fertigstellung des Projekts innerhalb des vorgegebenen Zeitrahmens verhinderten. Trotz dieser Schwierigkeiten war das Projekt eine bereichernde Erfahrung, insbesondere im Hinblick auf die Organisation und Durchführung von zukünftigen Projekten.

Es lässt sich festhalten, dass das Projekt trotz der Hindernisse eine wertvolle Lerngelegenheit bot, durch die die Teammitglieder nicht nur fachliche, sondern auch überfachliche Kompetenzen erweitern konnten. Diese Erfahrungen stellen eine wichtige Grundlage für die erfolgreiche Durchführung zukünftiger Projekte dar.


\subsubsection{LF04: Aufnahme von Audioquelle über REC IN}
\hypertarget{lf-audiorecord}{}

\begin{table}[h!]
	\begin{tabularx}{\textwidth}{|l|X|}
		\hline
		\textbf{Priorität} & Muss \\ \hline
		\textbf{Akteur} & Benutzer \\ \hline
		\textbf{Beschreibung} & Das System muss eine Audioquelle mit Line-Pegel, welche in der Eingangsbuchse eingesteckt ist aufnehmen können, und diese Aufnahme als korrekt kodierte PCM .wav Datei auf der SD-Karte abspeichern. Nach der Aufnahme muss eine Indexierung und Klassifizierung durch das Neuronale Netz geschehen (Wie in \textbf{\hyperlink{lf-nn-01}{LF03}} beschrieben)
		\\ \hline
	\end{tabularx}
\end{table}




\subsubsection{LF05: Wiedergabe von Audiodaten über OUT}
\hypertarget{lf-audioplayback}{}

\begin{table}[h!]
	\begin{tabularx}{\textwidth}{|l|X|}
		\hline
		\textbf{Priorität} & Muss \\ \hline
		\textbf{Akteur} & Benutzer \\ \hline
		\textbf{Beschreibung} & Das System muss Stereo PCM .wav Audiodaten, welche auf der SD-Karte gespeichert sind, über den Audiocodec abspielen können.
		
		Ziel ist eine zuverlässige, saubere Audiowiedergabe, die keine Störgeräusche wie Knackser und andere Artifakte produziert.
		\\ \hline
	\end{tabularx}
\end{table}



\subsubsection{LF06: Dynamische Änderung der Tonhöhe/Abspielgeschwindigkeit}
\hypertarget{lf-pitchaudio}{}

\begin{table}[h!]
	\begin{tabularx}{\textwidth}{|l|X|}
		\hline
		\textbf{Priorität} & Muss \\ \hline
		\textbf{Akteur} & Benutzer \\ \hline
		\textbf{Beschreibung} & Das System muss die Audiodaten (wie in \textbf{\hyperlink{lf-audioplayback}{LF05}} beschrieben) in verschiedenen Abspielgeschwindigkeiten wiedergeben können. So kann die Tonhöhe angepasst werden.
		
		Hierbei ist ein Algorithmus zu implementieren, der eine auditiv qualitative Tonhöhenanpassung umsetzt. Artifakte sollen vermieden werden.
		\\ \hline
	\end{tabularx}
\end{table}


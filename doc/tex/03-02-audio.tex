
\subsubsection{LFxx: Aufnahme von Audioquelle über REC IN}
\label{lf-audiorecord}

Das System muss eine Audioquelle mit Line-Pegel, welche in der Eingangsbuchse eingesteckt ist aufnehmen können, und diese Aufnahme als korrekt kodierte PCM .wav Datei auf der SD-Karte abspeichern. Nach der Aufnahme muss eine Indexierung und Klassifizierung durch das Neuronale Netz geschehen (Wie in LFxxx beschrieben)


\subsubsection{LFxx: Wiedergabe von Audiodaten über OUT}
\label{lf-audioplayback}

Das System muss Stereo PCM .wav Audiodaten, welche auf der SD-Karte gespeichert sind, über den Audiocodec abspielen können.

Ziel ist eine zuverlässige, saubere Audiowiedergabe, die keine Störgeräusche wie Knackser und andere Artifakte produziert.

\subsubsection{LFxx: Dynamische Änderung der Tonhöhe/Abspielgeschwindigkeit}
\label{lf-pitchaudio}

Das System muss die Audiodaten (wie in \textbf{\hyperlink{lf-audioplayback}{LFxx}} beschrieben) in verschiedenen Abspielgeschwindigkeiten wiedergeben können. So kann die Tonhöhe angepasst werden.

Hierbei ist ein Algorithmus zu implementieren, der eine auditiv qualitative Tonhöhenanpassung umsetzt. Artifakte sollen vermieden werden.
\subsection{Performance}

\begin{table}[h!]
	\begin{tabularx}{\textwidth}{|l|X|}
		\hline
		\textbf{/LQ1/} & \textbf{Latenz} \\ \hline
		\textbf{Kategorie} & Performance \\ \hline
		\textbf{Beschreibung} & Die Latenzzeit zwischen der Aufnahme eines Audio-Samples und dessen Wiedergabe sollte minimal sein (z.B. <10ms), um eine reibungslose Benutzererfahrung zu gewährleisten. \\ \hline
	\end{tabularx}
\end{table}

\begin{table}[h!]
	\begin{tabularx}{\textwidth}{|l|X|}
		\hline
		\textbf{/LQ2/} & \textbf{Echtzeitverarbeitung} \\ \hline
		\textbf{Kategorie} & Performance \\ \hline
		\textbf{Beschreibung} & Das System muss in der Lage sein, Audio-Daten in Echtzeit zu verarbeiten, um eine sofortige Rückmeldung und Sampleklassifizierung zu ermöglichen. \\ \hline
	\end{tabularx}
\end{table}

\begin{table}[h!]
	\begin{tabularx}{\textwidth}{|l|X|}
		\hline
		\textbf{/LQ3/} & \textbf{Datenübertragungsrate} \\ \hline
		\textbf{Kategorie} & Performance \\ \hline
		\textbf{Beschreibung} & Die Middleware muss hohe Datenübertragungsraten unterstützen, um eine kontinuierliche und verlustfreie Audioübertragung zu gewährleisten. \\ \hline
	\end{tabularx}
\end{table}

\subsection{Benutzerfreundlichkeit}

\begin{table}[h!]
	\begin{tabularx}{\textwidth}{|l|X|}
		\hline
		\textbf{/LQ4/} & \textbf{Intuitive Benutzeroberfläche} \\ \hline
		\textbf{Kategorie} & Benutzerfreundlichkeit \\ \hline
		\textbf{Beschreibung} & Die Benutzeroberfläche muss intuitiv und leicht bedienbar sein, damit Nutzer ohne umfangreiche Schulung schnell damit arbeiten können. \\ \hline
	\end{tabularx}
\end{table}

\begin{table}[h!]
	\begin{tabularx}{\textwidth}{|l|X|}
		\hline
		\textbf{/LQ5/} & \textbf{Feedback-Mechanismen} \\ \hline
		\textbf{Kategorie} & Benutzerfreundlichkeit \\ \hline
		\textbf{Beschreibung} & Das System sollte klares und sofortiges Feedback auf Benutzeraktionen geben, um Fehlbedienungen zu minimieren. \\ \hline
	\end{tabularx}
\end{table}

\newpage
\subsection{Effizienz}

\begin{table}[h!]
	\begin{tabularx}{\textwidth}{|l|X|}
		\hline
		\textbf{/LQ6/} & \textbf{Ressourcennutzung} \\ \hline
		\textbf{Kategorie} & Effizienz \\ \hline
		\textbf{Beschreibung} & Das System sollte sparsam mit den verfügbaren Ressourcen (CPU, Speicher, Energie) umgehen, um eine lange Betriebsdauer und Stabilität zu gewährleisten. \\ \hline
	\end{tabularx}
\end{table}

\begin{table}[h!]
	\begin{tabularx}{\textwidth}{|l|X|}
		\hline
		\textbf{/LQ7/} & \textbf{Optimierung der Audioverarbeitung} \\ \hline
		\textbf{Kategorie} & Effizienz \\ \hline
		\textbf{Beschreibung} & Algorithmen für die Audioverarbeitung und Klassifizierung sollten so optimiert sein, dass sie minimale Rechenressourcen benötigen, ohne die Genauigkeit zu beeinträchtigen. \\ \hline
	\end{tabularx}
\end{table}